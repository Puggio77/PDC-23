%Document format
\documentclass[a4paper,12pt]{article}

%Libraries
\usepackage{makeidx} %per l'indice
\usepackage{graphicx} %per le immagini
\usepackage[italian]{babel}

%preamble
\author{R. Puggioni, R. Regina, F. Trotti}
\title{TITOLO}
\makeindex

%main
\begin{document}
\maketitle

\clearpage
\printindex
\clearpage

\tableofcontents

\newpage

\section{Descrizione del problema}
\textit{Sviluppare un algoritmo per il calcolo della somma di N numeri reali, in ambiente di calcolo parallelo su architettura MIMD a memoria distribuita, che utilizzi la Liberia MPI.}





In un ambiente di calcolo parallelo la somma di N numeri deve essere prima strutturato in un certo modo. Per poter risolvere tale problema bisogna innanzitutto decomporre il nostro problema di dimensione N in P sottoproblemi di dimensione N/P e risolvi conte su più calcolatori. 
Per la somma di N numeri in ambiente di calcolo parallelo dobbiamo suddividere la somma in somme parziali ed assegnare ciascuna somma parziale ad un processore per poi ricombinare le somme parziali e ottenere la somma totale+

\textit{L'Algoritmo deve: \\ \\
1)Pendere in input il numero N di numeri da sommare \\ \\
2) Prendere in input i numeri se N < = 20\\ \\
3) Generare i numeri random se N > 2 \\ \\
4) Prendere in input il numeri di strategia da utilizzare \\ \\
5)Implementare la I, la II e la III strategia di comunicazione \\ \\
}




\section{descrizione della strategia di parallelizzazione implementata}

Per l'algoritmo della somma di N numeri possiamo applicare tre tipi di  strategie differenti:

\subsection{Strategia I}

In questo tipo di strategia ogni processore che andiamo a utilizzare calcola la propria somma parziale, ad ogni passo  avremo che ciascun processore invia tale valore ad unico processore prestabilito. Nel nostro processore conteniamo la somma totale

\subsection{Strategia II}

Nella seconda strategia come nella prima ogni strategia calcola la propria somma parziale. Coppie distinte di processori comunicano contemporaneamente ad ogni passo. 


\subsection{Strategia III}

In questa terza strategia il risultato della somma di N n


\section{Codice}



\section{Esempio d'uso degli input presenti nel software}


\section{Sezione dei grafici}
\subsection{tempo d'esecuzione}
\subsection{speed-up}
\subsection{efficienza}



\end{document}
